\chapter{Zusammenfassung}
\label{cha:zusammenfassung}

{\tiny Auf zwei bis drei Seiten soll auf folgende Punkte eingegangen werden:

\begin{itemize}
	\item Welches Ziel sollte erreicht werden
	\item Welches Vorgehen wurde gewählt
	\item Was wurde erreicht, zentrale Ergebnisse nennen, am besten quantitative Angaben machen
	\item Konnten die Ergebnisse nach kritischer Bewertung zum Erreichen des Ziels oder zur Problemlösung beitragen
	\item  Ausblick
\end{itemize}

In der Zusammenfassung sind unbedingt klare Aussagen zum Ergebnis der Arbeit zu nennen. Üblicherweise können Ergebnisse nicht nur qualitativ, sondern auch quantitativ benannt werden, z.~B. \glqq \ldots konnte eine Effizienzsteigerung von \SI{12}{\percent} erreicht werden.\grqq~oder \glqq \ldots konnte die Prüfdauer um \SI{2}{\hour} verkürzt werden\grqq.

Die Ergebnisse in der Zusammenfassung sollten selbstverständlich einen Bezug zu den in der Einleitung aufgeführten Fragestellungen und Zielen haben.}


Die Kombination aus WMS, ERP und den automatisierten Regalbediengeräten schafft ein vollständig integriertes, „smartes“ Lager. Alle Prozesse laufen digital gesteuert ab, was nicht nur die Geschwindigkeit und Präzision der Lagerprozesse erhöht, sondern auch die Datentransparenz und Steuerbarkeit im gesamten Unternehmen verbessert.