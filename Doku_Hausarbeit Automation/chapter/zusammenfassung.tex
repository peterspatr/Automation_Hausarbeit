\chapter{Zusammenfassung}

Die Kombination aus WMS, ERP und den automatisierten Regalbediengeräten schafft ein vollständig integriertes, „smartes“ Lager. Alle Prozesse laufen digital gesteuert ab, was nicht nur die Geschwindigkeit und Präzision der Lagerprozesse erhöht, sondern auch die Datentransparenz und Steuerbarkeit im gesamten Unternehmen verbessert.
Die Automatisierung des Hochregallagers bietet eine erhebliche Chance, die Effizienz zu steigern und die Betriebskosten nachhaltig zu senken. Der derzeitige manuelle Prozess mit Gabelstaplern und händischer Dateneingabe verursacht hohe Personalkosten und Fehlerkosten. Durch die Implementierung eines automatisierten Systems, das Regalbediengeräte, Förderbänder und ein Lagerverwaltungssystem umfasst, werden diese ineffizienten Prozesse deutlich reduziert.

Die jährlichen Einsparungen von etwa 380.000 € durch geringeren Personalaufwand und reduzierte Fehler- und Betriebskosten ermöglichen es, die Investitionskosten von 2.025.000 € in weniger als 5,5 Jahren zu amortisieren. Nach dieser Amortisationszeit wird das Unternehmen langfristig von den Einsparungen profitieren. Darüber hinaus erhöht die Automatisierung nicht nur die Effizienz, sondern auch die Sicherheit, da der Einsatz von Gabelstaplern und damit das Unfallrisiko minimiert wird.

Die Investition in die Automatisierung ist daher eine wirtschaftlich vorteilhafte Entscheidung, die sowohl die Wettbewerbsfähigkeit steigert als auch zu einer erheblichen Kostenersparnis führt.