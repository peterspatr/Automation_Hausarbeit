\chapter{Stakeholder}
Die Automatisierung eines Hochregallagers ist ein komplexes Unterfangen, das eine Vielzahl von Interessengruppen betrifft. Es gibt everschiedene Gruppen von Stakeholdern. 
Zu den internen Stakeholdern zählen die Mitarbeiter des Unternehmens, welche direkt im Lagerbetrieb tätig sind. \\
Auch das Management und die Führungskräfte sind wichtige interne Stakeholder, da sie die strategischen Entscheidungen treffen und die Verantwortung für den Projekterfolg tragen \autocite{trusteddecisions_stakeholderanalyse_2024}.

Zu den externen Stakeholdern zählen unter anderem Kunden, die von einer verbesserten Effizienz und Zuverlässigkeit des Lagers profitieren können, sowie Lieferanten, deren Prozesse sich möglicherweise an das neuen automatisierten System anpassen muss.
unter anderem Kunden, die von einer verbesserten Effizienz und Zuverlässigkeit des Lagers profitieren können, sowie Lieferanten, deren Prozesse sich möglicherweise an die neuen automatisierten Systeme anpassen müssen. Auch die Technologiepartner und Systemintegratoren, die an Umsetzung beteilig sind, zählen zu den externen Stakeholdern \autocite{trusteddecisions_stakeholderanalyse_2024} \autocite{redaktion_stakeholder-management_2013}.

Das Top-Management oder die wichtige Kunden, sollten besonders eng in den Prozess eingebunden werden. Stakeholder mit geringerem Einfluss, aber hohem Interesse, wie etwa die Lagerarbeiter, sollten aktiv informiert und beteiligt werden \autocite{trusteddecisions_stakeholderanalyse_2024}.



\chapter{Zusammenfassung}

Die Kombination aus WMS, ERP und den automatisierten Regalbediengeräten schafft ein vollständig integriertes, „smartes“ Lager. Alle Prozesse laufen digital gesteuert ab, was nicht nur die Geschwindigkeit und Präzision der Lagerprozesse erhöht, sondern auch die Datentransparenz und Steuerbarkeit im gesamten Unternehmen verbessert.
Die Automatisierung des Hochregallagers bietet eine erhebliche Chance, die Effizienz zu steigern und die Betriebskosten nachhaltig zu senken. Der derzeitige manuelle Prozess mit Gabelstaplern und händischer Dateneingabe verursacht hohe Personalkosten und Fehlerkosten. Durch die Implementierung eines automatisierten Systems, das Regalbediengeräte, Förderbänder und ein Lagerverwaltungssystem umfasst, werden diese ineffizienten Prozesse deutlich reduziert.

Die Umsetzung des Automatisierungsprojekts ist auf einen Zeitraum von etwa 12 Monaten ausgelegt. Der Übergang erfolgt schrittweise, um den laufenden Betrieb durchgehend sicherzustellen.

Die jährlichen Einsparungen von etwa 450.000 € durch geringeren Personalaufwand und reduzierte Fehler- und Betriebskosten ermöglichen es, die Investitionskosten von 1.610.000 € in weniger als 3,6 Jahren zu amortisieren. Nach dieser Amortisationszeit wird das Unternehmen langfristig von den Einsparungen profitieren. Darüber hinaus erhöht die Automatisierung nicht nur die Effizienz, sondern auch die Sicherheit, da der Einsatz von Gabelstaplern und damit das Unfallrisiko minimiert wird.

Die Investition in die Automatisierung ist daher eine wirtschaftlich vorteilhafte Entscheidung, die sowohl die Wettbewerbsfähigkeit steigert als auch zu einer erheblichen Kostenersparnis führt.




