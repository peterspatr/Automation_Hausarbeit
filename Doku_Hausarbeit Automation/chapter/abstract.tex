\chapter*{Kurzfassung} %*-Variante sorgt dafür, das Abstract nicht im Inhaltsverzeichnis auftaucht

%*******************************************************************************************************************

%In dieser Hausarbeit wird einer fiktiven Optimierung eines Hochregallagers nachgegangen. Aufgrund des Fachkräftemangels, auch im Bereich Logistik, wird es immer wichtiger, möglich viele Arbeitsschritte zu automatisieren.
%Momentan befindet sich das Hochregal in einem kompletten manuellen Betrieb. Die Ware muss von einem Mitarbeiter händisch einsortiert werden. Dieser pflegt auch die Daten ins System ein und bestellt bei Bedarf die Ware nach.
%All diese Arbeitsschritte sind zeitintensiv und anfällig für Fehler.
%-----------------------------------Problemstellung---------------------------------------------------------------
%Ziel ist es, durch eine Automatisierung soll das Hochregallager effizienter werden. Die Automatisierung soll zu einer höheren Produktivität führen, indem die Bearbeitungszeit verringert und die Fehler minimiert werden. Dafür soll das Hochregalsystem in der Lage sein, die Waren eigenständig ein- und auslagern zu können. 
%Bei einem niedrigen Lagerbestand soll über das ERP-System neue Ware nachbestellt werden. Hierfür ist es notwendig, dass das Hochregallager den Lagerbestand in Echtzeit überwacht. Um diese Anforderung umzusetzen, ist es wichtig, dass alle Produkte zuverlässig erkannt werden. Die Erkennung erfolgt über die RFID-Technologie, welche eine hohe Digitalisierung des Lagers voraussetzt, um die Informationen kontinuierlich aktualisieren zu können.
%-----------------------------------Zielsetzung---------------------------------------------------------------------
%% Der Absatz gefällt mir noch nicht 
%Ziel dieser Hausarbeit ist es, ein umfassendes Grobkonzept zu entwickeln. Dieses Konzept soll dabei helfen, die Idee eines automatisierten Hochregallagers klar und anschaulich der Geschäftsleitung zu präsentieren. 
%Das Aufzeigen der Vorteile soll die Manager von einer Umstellung auf eine %Automation überzeugen und die Bereitstellung eines Budgets sichern.
Ziel dieser Hausarbeit ist es, ein umfassendes Grobkonzept für die Automatisierung eines Hochregallagers zu entwickeln. Dieses Konzept soll eine detaillierte und klare Darstellung der erforderlichen Schritte, Technologien und Vorteile der Automatisierung bieten, um der Geschäftsleitung eine fundierte Entscheidungsgrundlage zu liefern. Dabei werden sowohl technische als auch wirtschaftliche Aspekte berücksichtigt, um das Potenzial der Automatisierung zur Effizienzsteigerung und Kostensenkung aufzuzeigen.

Ein wesentlicher Bestandteil des Konzepts ist es, die Geschäftsleitung von den konkreten Vorteilen der Umstellung zu überzeugen. Hierbei wird insbesondere auf die Reduzierung von Fehlern, die Verbesserung der Lagerprozesse und die langfristige Kostensenkung eingegangen. Zudem wird verdeutlicht, wie die Automatisierung die Produktivität steigern kann, indem sie manuelle und zeitaufwendige Tätigkeiten durch effizientere, automatisierte Abläufe ersetzt.

Die Präsentation dieser Vorteile hat das Ziel, die Entscheidungsträger von der Notwendigkeit und Wirtschaftlichkeit des Projekts zu überzeugen und die Bereitstellung eines Budgets für die Realisierung des automatisierten Systems zu sichern. Durch eine klare Darstellung der zukünftigen Einsparpotenziale und der positiven Auswirkungen auf den Betrieb soll das Konzept eine solide Grundlage für die erfolgreiche Umsetzung der Automatisierung im Hochregallager schaffen.


\chapter*{Abstract} %*-Variante sorgt dafür, das Abstract nicht im Inhaltsverzeichnis auftaucht
The goal of this paper is to develop a comprehensive rough concept for the automation of a high-bay warehouse. This concept should provide a detailed and clear representation of the necessary steps, technologies, and benefits of automation to give the management a well-founded decision-making basis. Both technical and economic aspects are considered in order to highlight the potential of automation to increase efficiency and reduce costs.

A key component of the concept is to convince the management of the specific advantages of the transition. The focus is particularly on reducing errors, improving warehouse processes, and achieving long-term cost savings. It will also be demonstrated how automation can increase productivity by replacing manual and time-consuming tasks with more efficient, automated processes.

Presenting these advantages aims to convince decision-makers of the necessity and economic viability of the project and secure the allocation of a budget for the implementation of the automated system. By clearly outlining the future savings potential and the positive impact on operations, the concept should create a solid foundation for the successful implementation of automation in the high-bay warehouse.
\clearpage