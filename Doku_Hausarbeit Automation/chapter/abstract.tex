\chapter*{Kurzfassung} %*-Variante sorgt dafür, das Abstract nicht im Inhaltsverzeichnis auftaucht

%*******************************************************************************************************************
Aufgrund des Fachkräftemangels, auch im Bereich Logistik, wird es immer wichtiger, möglich viele Arbeitsschritte zu automatisieren.
In dieser Hausarbeit wird einer fiktiven Optimierung eines Hochregallagers nachgegangen. 
Momentan befindet sich das Hochregal in einem kompletten manuellen Betrieb. Die Ware muss von einem Mitarbeiter händisch einsortiert werden. Dieser pflegt auch die Daten ins System ein und bestellt bei Bedarf die Ware nach.
All diese Arbeitsschritte sind zeitintensiv und anfällig für Fehler.
%-----------------------------------Problemstellung---------------------------------------------------------------
Ziel ist es, durch eine Automatisierung soll das Hochregallager effizienter werden. Die Automatisierung soll zu einer höheren Produktivität führen, indem die Bearbeitungszeit verringert und die Fehler minimiert werden. Dafür soll das Hochregalsystem in der Lage sein, die Waren eigenständig ein- und auslagern zu können. 
Bei einem niedrigen Lagerbestand soll über das ERP-System neue Ware nachbestellt werden. Hierfür ist es notwendig, dass das Hochregallager den Lagerbestand in Echtzeit überwacht. Um diese Anforderung umzusetzen, ist es wichtig, dass alle Produkte zuverlässig erkannt werden. Die Erkennung erfolgt über die RFID-Technologie, welche eine hohe Digitalisierung des Lagers voraussetzt, um die Informationen kontinuierlich aktualisieren zu können.
%-----------------------------------Zielsetzung---------------------------------------------------------------------
%% Der Absatz gefällt mir noch nicht 
Ziel dieser Hausarbeit ist es, ein umfassendes Grobkonzept zu entwickeln. Dieses Konzept soll dabei helfen, die Idee eines automatisierten Hochregallagers klar und anschaulich der Geschäftsleitung zu präsentieren. 
Das Aufzeigen der Vorteile soll die Manager von einer Umstellung auf eine Automation überzeugen und die Bereitstellung eines Budgets sichern.



\chapter*{Abstract} %*-Variante sorgt dafür, das Abstract nicht im Inhaltsverzeichnis auftaucht

English translation of the \glqq Kurzfassung\grqq.

\clearpage