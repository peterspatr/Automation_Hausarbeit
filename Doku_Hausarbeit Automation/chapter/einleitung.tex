\chapter{Einleitung}

Sehr geehrte Geschäftsführung,  

die Automatisierung unseres Hochregallagers bietet eine einzigartige Möglichkeit, die Effizienz und Wettbewerbsfähigkeit unseres Unternehmens maßgeblich zu steigern. Mit modernen Technologien können wir Bearbeitungszeiten verkürzen, Fehlerquoten minimieren und die Sicherheit unserer Mitarbeiter erhöhen, während wir gleichzeitig langfristige Kosten senken. Dieses Projekt ist eine zukunftsweisende Investition, die uns flexibler und nachhaltiger auf die Anforderungen eines dynamischen Marktes reagieren lässt.  

Für die Erstellung der vorliegenden Dokumentation haben zahlreiche Experten aus den Bereichen Logistik, Automatisierungstechnik und IT eng zusammengearbeitet. Ihre Fachkenntnisse und tiefgehenden Analysen haben zu einem umfassenden Verständnis der Chancen und Herausforderungen dieses Projekts geführt. Auf den folgenden Seiten finden Sie eine kurze präzise Darstellung der Vorteile, die erforderlichen Maßnahmen und Vorgehensweise, die kalkulierten Kosten, aber auch die Risiken dieses Projekts.  
Das Automatsierungsteam ladet Sie herzlich ein, dieses Wissen zu nutzen und mit uns gemeinsam den nächsten Schritt in eine innovative und erfolgreiche Zukunft zu gehen. 

Bei Rückfragen, Anregungen oder weiteren Informationen steht das Automatisierungsteam gerne zur Verfügung. Technische Unterlagen, eingeholte Angebote und weitere Details können auf Anfrage bereitgestellt werden. Zögern Sie nicht, sich mit uns in Verbindung zu setzen, um zusätzliche Einblicke in das Projekt zu erhalten. Wir freuen uns auf einen offenen Austausch und stehen für jegliche Unterstützung bereit.


\chapter{Ausgangssituation} 

Das Hochregallager unseres Unternehmens besteht aus insgesamt 8 Regalreihen mit jeweils 11 Ebenen und 10 Stellplätzen pro Ebene, was einer Gesamtkapazität von 880 Palettenplätzen entspricht. Der gesamte Lagerbetrieb erfolgt derzeit manuell und ist stark von den menschlichen Arbeitskräften abhängig.
\begin{itemize}
	\item \textbf{Kapazität und Arbeitsabläufe:} 
	Der tägliche Warendurchsatz beträgt etwa 250 Paletten im Wareneingang und 300 Paletten im Warenausgang. Die Paletten werden ausschließlich mit Gabelstaplern transportiert, wobei die Ein- und Auslagerung viel Zeit in Anspruch nimmt. Der Prozess der Warenbewegung wird anschließend manuell im Computersystem erfasst, was oft zu Verzögerungen und Fehlern bei der Dateneingabe führt.
	
	\item \textbf{Mitarbeiter und Arbeitszeiten:}
	Der Betrieb erfordert den Einsatz von 5 Mitarbeitern pro Schicht, die im Zwei-Schicht-System arbeiten und somit täglich 16 Stunden abdecken. Zu den Hauptaufgaben der Mitarbeiter gehören das Fahren der Gabelstapler, die Organisation der Warenbewegungen und die manuelle Dokumentation im Lagerverwaltungssystem. Diese Tätigkeiten sind sowohl körperlich belastend als auch zeitintensiv.
\end{itemize}


\subsection*{Herausforderungen im aktuellen Zustand}

\begin{itemize}
	\item \textbf{Begrenzte Kapazität}: Mit 880 Palettenplätzen stößt das Lager häufig an seine Grenzen, insbesondere bei erhöhtem Warendurchsatz.
	\item \textbf{Fehleranfälligkeit}: Die manuelle Eingabe führt regelmäßig zu fehlerhaften Bestandsdaten und erhöhtem Aufwand für Korrekturen.
	\item \textbf{Hoher Zeitaufwand}: Die Kombination aus Gabelstaplertransport und manueller Erfassung verlangsamt die Arbeitsprozesse erheblich.
	\item \textbf{Physische Belastung}: Die Mitarbeiter sind durch das Bewegen schwerer Paletten und die repetitive Arbeit stark beansprucht.
	\item \textbf{Sicherheitsrisiken}: Die intensive Nutzung von Gabelstaplern birgt ein erhöhtes Unfallrisiko für die Mitarbeiter.
\end{itemize}

\subsection*{Fazit}

Der aktuelle Zustand des Hochregallagers zeigt deutlichen Optimierungsbedarf. Die Prozesse sind ineffizient, fehleranfällig und nicht für ein steigendes Arbeitsvolumen ausgelegt. Eine Automatisierung würde nicht nur die Arbeitsabläufe erheblich verbessern, sondern auch die Sicherheit, Genauigkeit und Effizienz des Lagers langfristig sicherstellen.