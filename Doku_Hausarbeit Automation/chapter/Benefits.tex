\chapter{Benefits}
%Automatisierte Hochregallager (AHL) sind zunehmend im Logistik- und Lagerbereich anzutreffen, da sie eine Vielzahl von Vorteilen bieten, die sowohl die Effizienz als auch die Sicherheit in der Lagerhaltung verbessern. Die wichtigsten Benefits lassen sich in den folgenden Kategorien zusammenfassen:
%
%1. Effizienz und Produktivität
%
%Automatisierte Hochregallager optimieren den Materialfluss und minimieren die Zeiten für Be- und Entladung. Laut einer Studie von Duflou et al. (2012) erhöhen AHL die Lagerkapazität und reduzieren gleichzeitig die Kommissionierzeit erheblich. Durch den Einsatz von Technologien wie automatischen Fahrzeugen und Regalbediengeräten können Waren schneller und präziser bewegt werden, was zu einer Steigerung der Gesamteffizienz führt.
%
%2. Platzersparnis
%
%Ein weiterer wesentlicher Vorteil von AHL ist die effiziente Nutzung des verfügbaren Raums. Da diese Systeme vertikal arbeiten, wird die Höhe eines Lagers besser ausgenutzt. Laut einem Bericht von Frazelle (2002) können durch den Einsatz von Hochregallagern die Lagerflächen um bis zu 80 % maximiert werden, da die Regalsysteme in der Höhe angeordnet werden.
%
%3. Verbesserte Sicherheit
%
%Die Automatisierung von Lagerprozessen verbessert die Sicherheitsstandards erheblich. Durch den Einsatz automatisierter Systeme wird die Notwendigkeit menschlicher Eingriffe minimiert, was das Risiko von Arbeitsunfällen verringert. Research von Kelle et al. (2019) zeigt, dass die Implementierung von automatisierten Lagerlösungen die Unfallrate im Vergleich zu traditionellen Lagersystemen signifikant senken kann.
%
%4. Kostenreduktion
%
%Langfristig führt die Automatisierung in Hochregallagern zu Kostensenkungen. Nach einer Untersuchung von Waller & Fawcett (2013) reduzieren Unternehmen ihre Betriebskosten um bis zu 30 % durch reduzierte Arbeitskosten und geringeren Materialverschleiß. Außerdem gestalten sich die Lagerbetriebskosten effizienter, da Automatiken im Vergleich zu manuellen Systemen schneller auf sich ändernde Lageranforderungen reagieren können.
%
%5. Echtzeit-Überwachung und -Management
%
%Die vielen Daten, die durch automatisierte Systeme generiert werden, ermöglichen eine präzise Analyse und eine optimierte Lagerbewirtschaftung. Mit fortschrittlichen Lagerverwaltungssystemen können Unternehmen in Echtzeit auf Bestandsveränderungen reagieren. Laut einer Studie von Gunasekaran et al. (2015) können Unternehmen durch Echtzeit-Datenanalysen schneller Entscheidungen treffen und somit ihre Reaktionszeiten auf Marktveränderungen verbessern.
%
%6. Flexibilität und Skalierbarkeit
%
%Automatisierte Hochregallager bieten eine hohe Flexibilität, um auf unterschiedliche Lagerbedarfe zu reagieren. Unternehmen können ihre Systeme anpassen oder erweitern, um saisonale Schwankungen und Veränderungen in der Nachfrage zu berücksichtigen. Studien von Rasid et al. (2018) zeigen, dass Betriebe, die automatisierte Lagereinheiten nutzen, ihre Betriebskapazität problemlos anpassen können, ohne signifikante Umstrukturierungen vorzunehmen.

%-------------------------------------------------------------------------------------------------
Im Kapitel Benefits, geht es um die verschieden Vorteile, die durch eine Automatiersierung eines Hochregallagers erzielt werden können
\begin{itemize}
	%
	\item \textbf{Platzersparnis:}
	Durch effiziente Nutzung des verfügbaren Raums. Durch die bessere Ausnutzung der Höhe können bei begrenztem Platz mehr Produkte im vertikal arbeitenden System gelagert werden. Ebenso können auch die Korridore minimiert werden, wodurch die Lagerkapazität weiter gesteigert werden kann.
	\autocite{frazelle2002}
	%
	\item \textbf{Bessere Bestandskontrolle und -übersicht:}
	Eine präzise Analyse und optimale Lagerbewirtschaftung werden durch die vom System generierten Daten ermöglicht. Mithilfe der Echtzeit-Datenanalyse lässt sich schneller Veränderungen, z.B. Marktveränderungen, Preisveränderungen reagieren. \autocite{Gunasekaran2015}
	%
	\item \textbf{höhere Arbeitssicherheit:}
	Aufgrund der Automatisierung verbessert sich die Arbeitssicherheit deutlich. Durch das Automatisieren der Prozesse wird das menschliche Eingreifen minimiert, wodurch sich auch das Risiko für Arbeitsunfälle verringert. Auch müssen die Mitarbeiter keine schweren Lasten mehr heben. \autocite{kelle2019safety}
	%
	\item \textbf{Kosteneinsparung:}
	Die Kosten für eine Umstellung auf ein automatisiertes System sind zwar recht kostenintensiv, allerdings amortisiert sich die Änderung meist innerhalb weniger Jahre. Grund dafür sind die reduzierten Personalkosten sowie die geringere Fehlerquote. Die effizientere Raumnutzung trägt ebenfalls positiv zur Kosteneinsparung bei. \autocite{mueller2023}
%---------------------------------------------------------------------------------------------
%	\item \textbf{Weniger Personalaufwand:}
%   \item \textbf{Erhöhte Lagerkapazität:}
	
\end{itemize}