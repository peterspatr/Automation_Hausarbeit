\chapter{Kostenkalkulation}
In diesem Abschnitt werden die einmaligen Investitionskosten für die Automatisierung unseres Hochregallagers (Tabelle 6.1) sowie die jährlichen Betriebskosten des aktuellen (Tabelle 6.3) und des zukünftigen Systems (Tabelle 6.2) gegenübergestellt. Ziel ist es, eine fundierte wirtschaftliche Bewertung vorzunehmen.  

Durch die Gegenüberstellung der Betriebskosten und die Berechnung der Amortisationszeit zeigt sich, ab wann sich die Investition in das neue automatisierte System für unser Unternehmen finanziell lohnt. Diese Analyse dient als Grundlage für die Entscheidungsfindung und unterstreicht die langfristigen Vorteile des Projekts.  
\section{Kostenkalkulation für die Automatisierung}
Tabelle 6.1 zeigt die einmaligen Investitionskosten für die Automatisierung des Hochregallagers. Die Kosten basieren auf Angeboten führender Anbieter, wobei das beste Preis-Leistungs-Verhältnis berücksichtigt wurde. Dabei wurden Hardware, Software-Integration und Schulungen sorgfältig ausgewählt, um den Anforderungen des Projekts zu entsprechen. Diese transparente Kalkulation bietet eine fundierte Basis für die wirtschaftliche Bewertung.
\begin{table}[H]
	\centering
	\begin{tabular}{|p{8cm}|p{4cm}|p{5cm}|}
		\hline
		\textbf{Kostenpunkt} & \textbf{Kosten (€)} & \textbf{Beschreibung} \\
		\hline
		Automatisierte Regalbediengeräte & 600.000 & 4 Geräte à 150.000 € \\
		\hline
		Fördertechniksysteme & 750.000 & 10 Meter Förderband à 75.000 € \\
		\hline
		Lagerverwaltungssystem (WMS) & 150.000 & Lizenz: 100.000 €, Installation: 50.000 € \\
		\hline
		Systemintegration & 80.000 & Einmalige Kosten für Integration und Tests \\
		\hline
		Schulung und Mitarbeitereinweisung & 10.000 & Training der Mitarbeiter \\
		\hline
		zusätzliche Sensorik & 20.000 & diverse Sensorik als Zusatzschutz \\
		\hline
		\textbf{Gesamtkosten (Einmalig)} & \textbf{1.610.000} & \\
		\hline
	\end{tabular}
	\caption{Kosten für die Automatisierung}
\end{table}

\section{jährliche Betriebskosten}
Die Tabelle 6.2 zeigt die jährlichen Betriebskosten von useren aktuellen Vorgehensweise. Diese Kosten fallen aufgrund des manuellen Betriebs vergleichsweise hoch aus.

\begin{table}[H]
	\centering
	\begin{tabular}{|p{8cm}|p{4cm}|p{5cm}|}
		\hline
		\textbf{Kostenpunkt} & \textbf{Kosten (€)} & \textbf{Beschreibung} \\
		\hline
		Personalkosten  & 500.000 & 10 Mitarbeiter, 50.000 €/Jahr pro Mitarbeiter \\
		\hline
		Betriebskosten Gabelstapler & 120.000 & Treibstoff, Wartung (4 Stapler à 30.000 €) \\
		\hline
		Fehlerkosten & 50.000 & Fehlbestände, manuelle Korrekturen \\
		\hline
		Unfallkosten & 30.000 & Direkte und indirekte Unfallfolgekosten \\
		\hline
		\textbf{Gesamtkosten} & \textbf{700.000} & \\
		\hline
	\end{tabular}
	\caption{Jährliche Betriebskosten im aktuellen Zustand}
\end{table}
Nach der Automatisierung, wie in Tabelle 6.3 dargestellt, sinken die Betriebskosten deutlich. Durch den Einsatz automatisierter Systeme reduzieren sich vor allem Personalkosten und Betriebsausgaben.

\begin{table}[H]
	\centering
	\begin{tabular}{|p{8cm}|p{4cm}|p{5cm}|}
		\hline
		\textbf{Kostenpunkt} & \textbf{Kosten (€)} & \textbf{Beschreibung} \\
		\hline
		Personalkosten & 200.000 & 4 Mitarbeiter, 50.000 €/Jahr pro Mitarbeiter \\
		\hline
		Wartungskosten  & 20.000 & Regelmäßige Wartung \\
		\hline
		Stromkosten Automatisierung & 30.000 & Betrieb der automatisierten Geräte \\
		\hline
		\textbf{Gesamtkosten} & \textbf{250.000} & \\
		\hline
	\end{tabular}
	\caption{Jährliche Betriebskosten des automatisierten Systems}
\end{table}

\section{Vergleich und Rentabilität}
In diesem Kapitel wird die Wirtschaftlichkeit der Automatisierung des Hochregallagers untersucht. Dabei werden die einmaligen Investitionskosten den jährlichen Betriebskosten gegenübergestellt, um die Einsparungspotenziale zu verdeutlichen. Zudem wird anhand der Amortisationszeit berechnet, wie lange es dauert, bis sich die Investition durch die reduzierten Betriebskosten rentiert. 

\textbf{Einmalige Investition:} 1.610.000 € \\

\textbf{Jährliche Betriebskosten:}
\begin{itemize}
	\item \textbf{Aktueller Zustand:} 700.000 €
	\item \textbf{Automatisiertes System:} 250.000 €
\end{itemize}

\textbf{Einsparung pro Jahr:}  
\[
\text{Einsparung} = 700.000 - 250.000 = 450.000 \, \text{€}
\]

\textbf{Amortisationszeit:}  
\[
\text{Amortisationszeit} = \frac{\text{Einmalige Investition}}{\text{Jährliche Einsparung}} = \frac{1.610.000}{450.000} = 3{,}58\, \text{Jahre}
\]

\subsection*{Fazit}
Das automatisierte System amortisiert sich in weniger als 4 Jahren und bietet langfristige Einsparungen. Es verbessert die Effizienz, reduziert Fehler und erhöht die Sicherheit, wodurch es eine zukunftsweisende Investition für das Unternehmen darstellt.


\chapter{Implementierungsplan}

Die Realisierung des Automatisierungsprojekts für das Hochregallager wird in einem Zeitraum von etwa 12 Monaten geplant. Dieser Zeitraum umfasst sämtliche Projektphasen von der Planung über die Implementierung bis hin zur Inbetriebnahme. Um eine möglichst reibungslose Einführung zu gewährleisten, wird das Projekt in klar definierte Meilensteine unterteilt.

\section{Projektzeitraum}

Der gesamte Projektzeitraum gliedert sich in folgende Phasen:

\begin{itemize}
\item \textbf{Planungs- und Konzeptionsphase (Monat 1-3):}
In dieser Phase werden die technischen Anforderungen spezifiziert, Anbieter ausgewählt und die endgültige Projektplanung abgeschlossen.
\item \textbf{Anschaffung und Installation (Monat 4-8):}
Nach der Bestellung der benötigten Komponenten wie Regalbediengeräte, Fördertechnik und Software wird deren Installation vor Ort durchgeführt. Parallel dazu wird das Lagerverwaltungssystem (WMS) integriert.
\item \textbf{Test- und Optimierungsphase (Monat 9-10):}
In dieser Phase werden alle Systeme auf ihre Funktionalität überprüft und gegebenenfalls angepasst. Testläufe werden durchgeführt, um sicherzustellen, dass die Automatisierung reibungslos funktioniert.
\item \textbf{Schulungsphase (Monat 10-11):}
Die Mitarbeiter werden im Umgang mit dem neuen System geschult, um eine nahtlose Nutzung zu gewährleisten.
\item \textbf{Inbetriebnahme (Monat 12):}
Das automatisierte Hochregallager wird offiziell in den Betrieb genommen.
\end{itemize}

\section{Übergangsmanagment}

Der Übergang vom manuellen zum automatisierten Betrieb wird sorgfältig geplant, um die laufenden Betriebsprozesse nicht zu unterbrechen. Während der Installations- und Testphase bleibt das manuelle System vollständig aktiv, sodass der tägliche Betrieb reibungslos fortgeführt werden kann. Die Einführung der Automatisierung erfolgt schrittweise, indem Teilbereiche nacheinander getestet und optimiert werden, um mögliche Störungen frühzeitig zu erkennen und zu beheben. Parallel dazu erhalten die Mitarbeiter eine umfassende Schulung, um sicher mit den neuen Systemen arbeiten zu können und die Akzeptanz der Automatisierung zu fördern. Nach der Inbetriebnahme wird das System in einer Monitoring-Phase überwacht, um sicherzustellen, dass alle Prozesse einwandfrei ablaufen. Durch diese schrittweise und durchdachte Vorgehensweise wird ein nahtloser Übergang ermöglicht