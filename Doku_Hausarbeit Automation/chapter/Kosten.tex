\chapter{Kosten}
\section*{Kostenkalkulation und Vergleich}
In diesem Abschnitt werden die einmaligen Investitionskosten für die Automatisierung unseres Hochregallagers (Tabelle 6.1) sowie die jährlichen Betriebskosten des aktuellen (Tabelle 6.3) und des zukünftigen Systems (Tabelle 6.2) gegenübergestellt. Ziel ist es, eine fundierte wirtschaftliche Bewertung vorzunehmen.  

Durch die Gegenüberstellung der Betriebskosten und die Berechnung der Amortisationszeit zeigt sich, ab wann sich die Investition in das neue automatisierte System für unser Unternehmen finanziell lohnt. Diese Analyse dient als Grundlage für die Entscheidungsfindung und unterstreicht die langfristigen Vorteile des Projekts.  
\subsection*{Kostenkalkulation für die Automatisierung}

\begin{table}[h!]
	\centering
	\begin{tabular}{|p{8cm}|p{4cm}|p{5cm}|}
		\hline
		\textbf{Kostenpunkt} & \textbf{Kosten (€)} & \textbf{Beschreibung} \\
		\hline
		Automatisierte Regalbediengeräte & 600.000 & 4 Geräte à 150.000 € \\
		\hline
		Fördertechniksysteme & 750.000 & 10 Meter Förderband à 75.000 € \\
		\hline
		Lagerverwaltungssystem (WMS) & 150.000 & Lizenz: 100.000 €, Installation: 50.000 € \\
		\hline
		Systemintegration & 80.000 & Einmalige Kosten für Integration und Tests \\
		\hline
		Schulung und Mitarbeitereinweisung & 10.000 & Training der Mitarbeiter \\
		\hline
		Wartungskosten & 20.000 & Regelmäßige Wartung \\
		\hline
		\textbf{Gesamtkosten (Einmalig)} & \textbf{1.610.000} & \\
		\hline
	\end{tabular}
	\caption{Kosten für die Automatisierung}
\end{table}


\begin{table}[h!]
	\centering
	\begin{tabular}{|p{8cm}|p{4cm}|p{5cm}|}
		\hline
		\textbf{Kostenpunkt} & \textbf{Kosten (€)} & \textbf{Beschreibung} \\
		\hline
		Personalkosten  & 300.000 & 6 Mitarbeiter, 50.000 €/Jahr pro Mitarbeiter \\
		\hline
		Betriebskosten Gabelstapler & 120.000 & Treibstoff, Wartung (4 Stapler à 30.000 €) \\
		\hline
		Fehlerkosten & 50.000 & Fehlbestände, manuelle Korrekturen \\
		\hline
		Unfallkosten & 30.000 & Direkte und indirekte Unfallfolgekosten \\
		\hline
		\textbf{Gesamtkosten} & \textbf{500.000} & \\
		\hline
	\end{tabular}
	\caption{Jährliche Betriebskosten im aktuellen Zustand}
\end{table}


\begin{table}[h!]
	\centering
	\begin{tabular}{|p{8cm}|p{4cm}|p{5cm}|}
		\hline
		\textbf{Kostenpunkt} & \textbf{Kosten (€)} & \textbf{Beschreibung} \\
		\hline
		Personalkosten (jährlich) & 100.000 & 2 Mitarbeiter, 50.000 €/Jahr pro Mitarbeiter \\
		\hline
		Wartungskosten (jährlich) & 20.000 & Regelmäßige Wartung \\
		\hline
		Stromkosten Automatisierung & 30.000 & Betrieb der automatisierten Geräte \\
		\hline
		\textbf{Gesamtkosten (jährlich)} & \textbf{150.000} & \\
		\hline
	\end{tabular}
	\caption{Jährliche Betriebskosten des automatisierten Systems}
\end{table}

\subsection*{Vergleich und Rentabilität}
\textbf{Einmalige Investition:} 1.610.000 € \\

\textbf{Jährliche Betriebskosten:}
\begin{itemize}
	\item \textbf{Aktueller Zustand:} 500.000 €
	\item \textbf{Automatisiertes System:} 150.000 €
\end{itemize}

\textbf{Einsparung pro Jahr:}  
\[
\text{Einsparung} = 500.000 - 150.000 = 350.000 \, \text{€}
\]

\textbf{Amortisationszeit:}  
\[
\text{Amortisationszeit} = \frac{\text{Einmalige Investition}}{\text{Jährliche Einsparung}} = \frac{1.610.000}{350.000} = 4{,}6 \, \text{Jahre}
\]

\subsection*{Fazit}
Das automatisierte System amortisiert sich in weniger als 5 Jahren und bietet langfristige Einsparungen. Es verbessert die Effizienz, reduziert Fehler und erhöht die Sicherheit, wodurch es eine zukunftsweisende Investition für das Unternehmen darstellt.


