\chapter{Risiken}


%Kapitel: Risiken beim Einsatz von Automatisierten Hochregallagern
%
%Automatisierte Hochregallager (AHL) haben in den letzten Jahren zunehmend an Bedeutung gewonnen, da sie die Effizienz in der Lagerlogistik erheblich steigern können. Sie zeichnen sich durch hohe Lagerdichte, schnellen Zugriff auf Waren und eine präzise Bestandsführung aus. Dennoch sind mit der Einführung und dem Betrieb solcher Systeme verschiedene Risiken verbunden. Diese Risiken betreffen sowohl technologische als auch organisatorische und sicherheitsrelevante Aspekte und können die Performance und Sicherheit des Lagersystems beeinträchtigen.
%1. Technologische Risiken
%
%Automatisierte Hochregallager beruhen auf einer komplexen Kombination von Hard- und Software, die für die Steuerung und den Betrieb des Lagers erforderlich ist. Ein technisches Versagen in einem dieser Bereiche kann zu erheblichen Störungen im Betriebsablauf führen.
%
%Systemausfälle und Softwarefehler: Ein häufiges Risiko bei der Einführung von automatisierten Systemen sind Softwarefehler oder Systemausfälle, die durch mangelnde Wartung, unzureichende Tests oder unvorhergesehene technische Probleme entstehen können. Im schlimmsten Fall kann dies zu Ausfällen im gesamten Lagerbetrieb führen, was hohe Kosten verursacht (Dürr et al., 2021).
%
%Inkompatibilitäten und Integration: Beim Aufbau eines automatisierten Hochregallagers werden häufig bestehende Systeme integriert. Inkompatibilitäten zwischen verschiedenen Softwarelösungen oder Hardwarekomponenten können zu Problemen führen, die die Effizienz des gesamten Systems beeinträchtigen (Hillebrand et al., 2020).
%
%2. Sicherheitsrisiken
%
%Sicherheitsrisiken betreffen sowohl die IT-Sicherheit als auch die physische Sicherheit der Mitarbeiter.
%%Cyberangriffe und Datenschutzverletzungen: Die zunehmende Vernetzung von Lager- und Logistiksystemen macht automatisierte Hochregallager anfällig für Cyberangriffe. Hackerangriffe können nicht nur zu einem Verlust von Geschäftsdaten führen, sondern auch die Steuerung der Logistikprozesse lahmlegen (Schneider, 2020). Daher sind robuste IT-Sicherheitsmaßnahmen unabdingbar, um das Risiko von Datenmissbrauch und Systemmanipulation zu minimieren.

%
%Unfälle und Verletzungen der Mitarbeiter: Auch wenn automatisierte Systeme darauf abzielen, den Menschen aus gefährlichen oder anstrengenden Arbeitsumgebungen zu befreien, gibt es immer noch Risiken im Umgang mit diesen Systemen. Unfälle können durch Fehler in der Programmierung oder durch unzureichend gesicherte Übergänge zwischen automatisierten und manuellen Prozessen auftreten (Bauer, 2021).
%
%3. Wirtschaftliche Risiken
%
%Die Implementierung eines automatisierten Hochregallagers ist mit erheblichen Investitionskosten verbunden. Neben den initialen Investitionen entstehen laufende Kosten für Wartung, Schulung und Systemaktualisierungen.
%
%Hohe Anfangsinvestitionen: Die Einführung eines automatisierten Lagersystems erfordert in der Regel hohe Kapitalaufwendungen für den Erwerb der Hardware, Software sowie für den Umbau bestehender Lagerstätten (Stark, 2019). Dies stellt vor allem für kleinere Unternehmen ein großes Risiko dar, wenn die Rendite nicht schnell genug realisiert werden kann.
%
%Kosten für Wartung und Reparaturen: Wie bei jeder komplexen Technologie sind regelmäßige Wartungsmaßnahmen erforderlich, um die Funktionsfähigkeit des Systems aufrechtzuerhalten. Verzögerungen oder Fehler bei der Wartung können die Produktivität mindern und zu zusätzlichen Kosten führen (Petersen & Lutz, 2020).
%
%4. Logistische Risiken
%
%Neben den technologischen und sicherheitsrelevanten Risiken gibt es auch logistische Risiken, die den Betrieb eines automatisierten Hochregallagers betreffen.
%
%Fehlende Flexibilität: Automatisierte Systeme sind oft weniger flexibel, wenn es um kurzfristige Änderungen in der Logistik oder bei unerwarteten Herausforderungen wie plötzlichen Nachfrageschwankungen geht. Eine Anpassung des Systems an neue Anforderungen erfordert häufig umfangreiche Änderungen, was mit zusätzlichen Kosten und Zeitaufwand verbunden ist (Schäfer & Müller, 2021).
%
%Störungen im Materialfluss: Der Materialfluss in einem automatisierten Lager wird durch die automatisierte Steuerung und den Einsatz von Robotern und Fördertechnik gesteuert. Fehler in der Steuerung oder Störungen in den mechanischen Systemen können den gesamten Materialfluss behindern, was zu Engpässen oder Verzögerungen in der Lieferung führt (Krause & Theis, 2018).
%
%5. Organisatorische Risiken
%
%Die Einführung eines automatisierten Hochregallagers erfordert eine gründliche Umstrukturierung der logistischen Prozesse und eine umfassende Schulung des Personals.
%
%Widerstand der Mitarbeiter: Ein organisatorisches Risiko stellt der Widerstand der Mitarbeiter dar, insbesondere wenn sie befürchten, dass ihre Arbeitsplätze durch die Automatisierung gefährdet sind. Ein schlechter Umgang mit diesen Ängsten und eine unzureichende Kommunikation können die Einführung eines automatisierten Systems behindern (Bauer & Mertens, 2022).
%
%Mangelnde Qualifikation der Mitarbeiter: Der Betrieb eines automatisierten Hochregallagers erfordert qualifizierte Fachkräfte, die die Technologie verstehen und bedienen können. Unzureichende Schulung oder ein Mangel an qualifizierten Arbeitskräften können die Effizienz des Systems beeinträchtigen und zu Fehlfunktionen führen (Becker, 2020).
%
%Fazit
%
%Der Einsatz von automatisierten Hochregallagern birgt zahlreiche Risiken, die von technologischen über sicherheitsrelevante bis hin zu wirtschaftlichen und organisatorischen Aspekten reichen. Um diese Risiken zu minimieren, ist eine gründliche Planung, die regelmäßige Wartung der Systeme und eine gute Schulung der Mitarbeiter unerlässlich. Durch eine transparente Kommunikation und die gezielte Einbindung der Mitarbeiter in den Veränderungsprozess können viele der organisatorischen Risiken gemindert werden. Langfristig können die Vorteile eines automatisierten Hochregallagers die potenziellen Risiken jedoch überwiegen, wenn diese Risiken entsprechend adressiert werden.
%Literaturverzeichnis
%
%Bauer, A. (2021). Sicherheit in automatisierten Lagersystemen. Springer Vieweg.
%Bauer, H., & Mertens, M. (2022). Digitale Transformation in der Logistik: Chancen und Herausforderungen. Wiley-VCH.
%Becker, M. (2020). Technologien der Logistik 4.0. Springer Gabler.
%Dürr, M., et al. (2021). Automatisierung in der Logistik: Grundlagen, Konzepte und Anwendungen. Hanser Verlag.
%Hillebrand, J., et al. (2020). Integration automatisierter Systeme in bestehende Logistikprozesse. Springer Vieweg.
%Krause, K., & Theis, D. (2018). Logistik 4.0 – Die digitale Revolution im Lager. Springer Gabler.
%Petersen, M., & Lutz, C. (2020). Wartung und Instandhaltung in der Industrie 4.0. Springer Vieweg.
%Schäfer, F., & Müller, S. (2021). Flexibilität und Effizienz in automatisierten Lagersystemen. Springer Gabler.
%Schneider, F. (2020). Cybersicherheit in der Logistik: Risiken und Lösungen. Springer Vieweg.
%Stark, A. (2019). Kostenrechnung in der Logistik: Grundlagen und Anwendungen. Springer Gabler.}
%***********************************************************************************************************************%***********************************************************************************************************************
In diesem Kapitel wird auf die möglichen Risiken näher eingegangen. Durch die Automatisierung eines Hochregallagers entstehen verschiedene Risiken.
Hier im Kapitel wird zuerst auf die technologischen Risiken eingegangen. Anschließend werden die Sicherheits- und wirtschaftlichen Risiken analysiert.
%
\section{Technologische Risiken}
Ein automatisiertes Hochregallager ist ein komplexes Zusammenspiel aus Software und Hardware. Das Auftreten eines technischen Problems in einem dieser Bereiche kann zu Störungen, was einen erheblichen Einfluss  auf den  Betriebsauflauf des Hochregallagers hat.
%
System- oder Softwarefehler stellen ein hohes Risiko dar. Sie könne durch unzureichenden Test oder durch mangelnde Wartung entstehen. Im schlimmsten Fall kann es dadurch zu Ausfällen oder Störungen im kompletten Lagerablauf kommen, was hohen Kosten verursachen kann \autocite{hartel_projektmanagement_2019}.\\
Bei einer späteren Erweiterung des Hochregallagers kann es zu Integrationsproblemen mit dem neuen System kommen. Um diesen Fehler zu vermeiden, muss sichergestellt werden, dass das automatisierte System Komponenten verwendet, welche in Zukunft kompatible mit einer voraussichtlichen Erweiterung sind. Optimalerweise wird hierfür bei der Auswahl auf Standardsysteme für die Industrie zurückgegriffen.
%--------------------------------------
\section{Sicherheitsrisiken}
Mit der zunehmenden Automatisierung und der mit verbunden Vernetzung des Lagersystems, auch mit externen Systemen steigt das Cyber-Risiko. Ein Cyber-Angriff auf das automatisierte Hochregallager, könnte nicht nur zum Verlust sämtlicher Daten führen, sonder auch den kompletten Lagerprozess stören und sogar lahmlegen. Daher sind robustere IT-Systeme notwendig, vor allem zu externen Schnittstellen, wie ERP-System \autocite{haumer_it-sicherheit_2015}.\\
%
Obwohl der Mensch durch die Automatisierung zunehmend ersetzt werden soll, kann es dennoch zu Gefährdungen kommen. Beinahe- oder Unfälle können durch Fehlfunktionen oder Fehlverhalten des Mitarbeiters auftreten. Um schwere Unfälle zu vermeiden müssen eindeutige Prozesse zum Stoppen des Systems definiert werden und es sind entsprechende Sicherheitsvorkehrungen bei der Planung zu berücksichtigt \autocite{fahl2016}.
%--------------------------------------
\section{Wirtschaftliche Risiken}
Der Umbau des Lagers auf ein automatisiertes System ist mit einer hohen Investition von Kapital verbunden. Da es einige kostenintensive Neuanschaffungen wie z.B. Steuerung, Sensoren, Software etc. bedarf. Diese aufgewendeten Kosten müssen durch die Steigerung der Effizienz und Produktivität eingespart werden. Es besteht das Risiko, dass es mehr Zeit braucht, bis sich die Umrüstung amortisiert hat \autocite{schmitz1994}.\\
%
Ein weiteres Risiko sind die Wartungs- und Instandhaltungskosten. Um einen fehlerfreien und reibungslosen Ablauf des Systems sicherzustellen, ist es wichtig regelmäßige Wartungen zu machen und eventuelle Verschleißteile austauschen. Die Kosten für Support und Instandhaltung könne die Zeit bis zur Amortisierung deutlich verlängern. 