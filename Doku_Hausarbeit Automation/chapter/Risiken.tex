\chapter{Risiken}


Die Umsetzung des Projekts zur Automatisierung des Hochregallagers bringt zwar erhebliche Vorteile mit sich, jedoch sind auch einige Risiken zu berücksichtigen, die im Rahmen der Planung und Durchführung unbedingt beachtet werden sollten.

\begin{itemize}
	\item Hohe Investitionskosten und Budgetrisiko
	Die Anschaffung und Installation automatisierter Lagertechnologien wie Regalbediengeräte, Fördertechnik und Lagerverwaltungssysteme (WMS) erfordern eine beträchtliche Anfangsinvestition. Es besteht das Risiko, dass die tatsächlichen Kosten die geplanten Budgets überschreiten, sei es durch zusätzliche technische Anforderungen oder unerwartete Anpassungen während der Implementierung. Diese finanzielle Belastung könnte den kurzfristigen Cashflow des Unternehmens belasten.
	\item Technologische Abhängigkeit und Wartungskosten
	Mit der Einführung automatisierter Systeme wächst die Abhängigkeit von Technologien, die regelmäßig gewartet und bei Bedarf repariert werden müssen. Das Risiko besteht darin, dass Wartungs- und Reparaturkosten höher als erwartet ausfallen können und dass technische Ausfälle den Betrieb verlangsamen oder gar zum Stillstand bringen könnten. Ein langfristiges Wartungs- und Ersatzteilkonzept ist daher notwendig, um Ausfallzeiten zu minimieren.
	\item Schulung und Anpassung des Personals
	 Die Einführung neuer Technologien erfordert eine umfassende Schulung der Mitarbeiter, um sicherzustellen, dass sie die Systeme effektiv nutzen können. Hier besteht das Risiko, dass das Personal Schwierigkeiten bei der Anpassung an die neuen Prozesse hat oder dass umfangreiche Schulungen notwendig werden, die wiederum Zeit und Kosten verursachen. Mitarbeiter, die bisher im manuellen Lagerbetrieb tätig waren, könnten zudem Bedenken hinsichtlich ihrer zukünftigen Rolle und Arbeitsplatzsicherheit haben, was sich auf die Motivation und die Akzeptanz des Projekts auswirken könnte.
	\item Technische Herausforderungen und Integrationsprobleme
	Die Implementierung automatisierter Systeme kann auf technische Herausforderungen stoßen, insbesondere wenn das neue System mit bestehenden IT-Infrastrukturen und betrieblichen Prozessen integriert werden muss. Hier besteht das Risiko, dass die Systeme nicht reibungslos zusammenarbeiten oder dass unerwartete Schnittstellenprobleme auftreten, die die Effizienz beeinträchtigen und die Projektzeitpläne verzögern könnten.
	\item Anfälligkeit für Cybersecurity-Risiken
	Da automatisierte Systeme oft über digitale Netzwerke gesteuert und überwacht werden, könnten sie anfällig für Cyberangriffe sein. Hacker könnten versuchen, Zugang zu den Systemen zu erhalten und so den Betrieb des Hochregallagers zu stören oder Daten zu stehlen. Ein erhöhtes Risiko besteht insbesondere dann, wenn die IT-Sicherheitsstandards nicht auf dem neuesten Stand gehalten werden.
	\item Abhängigkeit von Lieferanten und Fachkräften
	Der Erfolg des Projekts hängt auch von zuverlässigen Technologieanbietern und qualifizierten Fachkräften ab, die die Systeme installieren und warten können. Engpässe oder Lieferverzögerungen bei den Herstellern der benötigten Technologien könnten den Projektverlauf verzögern. Zudem ist der Zugang zu qualifizierten Technikern und IT-Experten entscheidend, um die Anlagen effektiv betreiben und warten zu können.
\end{itemize}
Um diese Risiken zu minimieren, sind sorgfältige Planung, ein Pufferbudget und ein robustes Risikomanagement erforderlich. Eine frühzeitige Schulung der Mitarbeiter und regelmäßige Sicherheitsupdates können ebenso dazu beitragen, potenzielle Probleme frühzeitig zu erkennen und zu lösen. Indem diese Risiken im Vorfeld identifiziert und mit geeigneten Maßnahmen adressiert werden, können wir die Erfolgschancen des Projekts maximieren und die Vorteile der Automatisierung langfristig sicherstellen.